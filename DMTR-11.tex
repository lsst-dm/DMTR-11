\documentclass[DM,toc,lsstdraft]{lsstdoc}
\usepackage{booktabs}

\title[Summer 2015 Characterization Report]{Characterization Metric Report: Science Pipelines Version 11.0 (Summer 2015)}
\author{Frossie Economou, John~Swinbank, Jim~Bosch, and Simon~Krughoff}
\date{2015-10-01}
\setDocRef{DMTR-11}

\setDocAbstract{%
Starting from Summer 2015, administrative ("cycle") releases are
accompanied by a measurements report characterising the current
performance. Metrics included in these reports are expected to increase
in number and sophistication at subsequent releases. This brief report
describe measurements of interest that were carried out.
}

\begin{document}
\maketitle

\section{Summary of Photometric Repeatability
Measurements}\label{summary-of-photometric-repeatability-measurements}

\emph{Submitted by Jim Bosch}

This dataset is a selection of $i$-band HyperSuprime-Cam engineering data
taken in the SDSS Stripe 82 region. This dataset consists of 30\,s
exposures, so it is somewhat similar to projected LSST data in depth.
Our current calibration approach has many limitations relative to what
we ultimately plan to implement for LSST:

\begin{itemize}
\item
  There's currently no relative calibration being run at all.
\item
  We have only limited correction for chromatic effects.
\item
  There's's currently no allowance for zeropoint variations smaller than
  the scale of a CCD.
\item
  We also use a much simpler sample selection than that proposed by the
  \SRD.
\end{itemize}

Annotated code to compute the metrics can be found at
\url{https://github.com/lsst/afw/blob/tickets/DM-3896/examples/repeatability.ipynb}.

\begin{small}
\begin{longtable}[]{@{}lllll@{}}
\toprule
\textbf{Metric Characterised} & \textbf{Metric} & \textbf{Target} &
\textbf{Measured} & \textbf{Measurement}\tabularnewline
 & \textbf{Ref} &  &
\textbf{Value} & \textbf{Method}\tabularnewline
\midrule
\endhead
Photometric repeatability (procCalRep) &
\href{https://jira.lsstcorp.org/browser/DLP-307}{DLP-307} & ≤ 13 mmag &
10.6 mmag & \href{https://jira.lsstcorp.org/browse/DM-3338}{DM-3338}
(using $i$-band data)\tabularnewline
Photometric repeatability (PA1gri) &
\href{https://jira.lsstcorp.org/browse/DLP-315}{DLP-315} & ≤ 13 mmag &
10.6 mmag &
\href{https://jira.lsstcorp.org/browse/DM-3338}{DM-3338}\tabularnewline
Photometric repeatability (PA1uzy) &
\href{https://jira.lsstcorp.org/browse/DLP-316}{DLP-316} & ≤ 13 mmag &
10.6 mmag & \href{https://jira.lsstcorp.org/browse/DM-3338}{DM-3338}
(using $i$-band data)\tabularnewline
\bottomrule
\end{longtable}
\end{small}

\section{Summary of Algorithmic Performance
Measurements}\label{summary-of-algorithmic-performance-measurements}

\emph{Submitted by John Swinbank}

The $i$-band HSC engineering data (described above) was used where
possible and the same caveats apply. Consult the tickets in the
Measurement Method column for more details.

\begin{small}
\begin{longtable}[]{@{}lllll@{}}
\toprule
\textbf{Metric Characterised} & \textbf{Metric} & \textbf{Target} &
\textbf{Measured} & \textbf{Measurement}\tabularnewline
 & \textbf{Ref} &  &
\textbf{Value} & \textbf{Method}\tabularnewline
\midrule
\endhead
Residual PSF Ellipticity Correlations (TE1) &
\href{https://jira.lsstcorp.org/browse/DLP-290}{DLP-290} & ≤ 5e-3 & 6e-5
&
\href{https://jira.lsstcorp.org/browse/DM-3040}{DM-3040}\tabularnewline
Residual PSF Ellipticity Correlations (TE2) &
\href{https://jira.lsstcorp.org/browse/DLP-290}{DLP-290} & ≤ 5e-3 & 2e-5
& \href{https://jira.lsstcorp.org/browse/DM-3047}{DM-3047
}\tabularnewline
Relative Astrometry (AM1) &
\href{https://jira.lsstcorp.org/browse/DLP-310}{DLP-310} & \textless{}
60 mas & 12.49 mas &
\href{https://jira.lsstcorp.org/browse/DM-3057}{DM-3057}\tabularnewline
Relative Astrometry (AM2) &
\href{https://jira.lsstcorp.org/browse/DLP-311}{DLP-311} & \textless{}
60 mas & 12.19 mas &
\href{https://jira.lsstcorp.org/browse/DM-3064}{DM-3064}\tabularnewline
\bottomrule
\end{longtable}
\end{small}

\section{Summary of Computational Performance
Measurements}\label{summary-of-computational-performance-measurements}

\emph{Submitted by John Swinbank and
Simon Krughoff}

At this point of Construction, the computational performance
measurements are a combination of precursor data processing and
extrapolation from R\&D assumptions.

DECam/HITS data was used for the OTT1 estimate and for the diffim and
single-frame measurement of the Alert Production Computational Budget in
combination with data from the
3rd Data Challenge \citedsp{Document-26217}.

For the Data Release Production of the computational budget, we used
DECam/HITS data for estimating diffim performance, HSC-I for assembling
and measuring coadds and for forced measurement, estimates from FDR for
multifit, and data from the 3rd Data Challenge for SDQA. Calculations
for the DRP computational budget used
\href{https://github.com/lsst-dm/kpm/blob/29c053f7b832e8bd999527e012681826fc0c201c/DLP-314:\%20DRP\%20Computational\%20Budget/LSST\%20DRP\%20Computational\%20Budget.ipynb}{this
iPython notebook}

\begin{small}
\begin{longtable}[]{@{}lllll@{}}
\toprule
\textbf{Metric Chracterised} & \textbf{Metric} & \textbf{Target} &
\textbf{Measured} & \textbf{Measurement}\tabularnewline
 & \textbf{Ref} &  &
\textbf{Value} & \textbf{Method}\tabularnewline
\midrule
\endhead
OTT1 & \href{https://jira.lsstcorp.org/browse/DLP-328}{DLP-328} & ≤ 240
sec & 200-250 sec &
\href{https://jira.lsstcorp.org/browse/DM-3724}{DM-3724}\tabularnewline
AP Computational Budget &
\href{https://jira.lsstcorp.org/browse/DLP-329}{DLP-329} & ≤ 231 TFLOPS
& 34-39 TFLOPS &
\href{https://jira.lsstcorp.org/browse/DM-3267}{DM-3267}\tabularnewline
DRP Computational Budget &
\href{https://jira.lsstcorp.org/browse/DLP-314}{DLP-314} & ≤ 645 TFLOPS
& 318 TFLOPS &
\href{https://jira.lsstcorp.org/browse/DM-3083}{DM-3083}\tabularnewline
\bottomrule
\end{longtable}
\end{small}

\bibliography{lsst}

\end{document}
